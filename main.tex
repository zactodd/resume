\documentclass[letterpaper,11pt]{article}

\usepackage{latexsym}
\usepackage[empty]{fullpage}
\usepackage{titlesec}
\usepackage{marvosym}
\usepackage[usenames,dvipsnames]{color}
\usepackage{verbatim}
\usepackage{enumitem}
\usepackage[hidelinks]{hyperref}
\usepackage{fancyhdr}
\usepackage[english]{babel}
\usepackage{tabularx}
\input{glyphtounicode}


%----------FONT OPTIONS----------
% sans-serif
% \usepackage[sfdefault]{FiraSans}
% \usepackage[sfdefault]{roboto}
% \usepackage[sfdefault]{noto-sans}
% \usepackage[default]{sourcesanspro}

% serif
% \usepackage{CormorantGaramond}
% \usepackage{charter}


\pagestyle{fancy}
\fancyhf{} % clear all header and footer fields
\fancyfoot{}
\renewcommand{\headrulewidth}{0pt}
\renewcommand{\footrulewidth}{0pt}

% Adjust margins
\addtolength{\oddsidemargin}{-0.5in}
\addtolength{\evensidemargin}{-0.5in}
\addtolength{\textwidth}{1in}
\addtolength{\topmargin}{-.5in}
\addtolength{\textheight}{1.0in}

\urlstyle{same}

\raggedbottom
\raggedright
\setlength{\tabcolsep}{0in}

% Sections formatting
\titleformat{\section}{
  \vspace{-4pt}\scshape\raggedright\large
}{}{0em}{}[\color{black}\titlerule \vspace{-5pt}]

% Ensure that generate pdf is machine readable/ATS parsable
\pdfgentounicode=1

%-------------------------
% Custom commands
\newcommand{\resumeItem}[1]{
  \item\small{
    {#1 \vspace{-2pt}}
  }
}



\newcommand{\resumeSubheading}[4]{
  \vspace{-2pt}
  \item
    \begin{tabular*}{0.97\textwidth}[t]{l@{\extracolsep{\fill}}r}
      \textbf{#1} & #2 \\
      \textit{\small#3} & \textit{\small #4} \\
    \end{tabular*}
    \vspace{-7pt}
}

\newcommand{\resumeSubSubheading}[2]{
    \item
    \begin{tabular*}{0.97\textwidth}{l@{\extracolsep{\fill}}r}
      \textit{\small#1} & \textit{\small #2} \\
    \end{tabular*}\vspace{-7pt}
}

\newcommand{\resumeProjectHeading}[2]{
    \item
    \begin{tabular*}{0.97\textwidth}{l@{\extracolsep{\fill}}r}
      \small#1 & #2 \\
    \end{tabular*}\vspace{-7pt}
}

\newcommand{\resumeSubItem}[1]{\resumeItem{#1}\vspace{-4pt}}

\renewcommand\labelitemii{$\vcenter{\hbox{\tiny$\bullet$}}$}

\newcommand{\resumeSubHeadingListStart}{\begin{itemize}[leftmargin=0.15in, label={}]}
\newcommand{\resumeSubHeadingListEnd}{\end{itemize}}
\newcommand{\resumeItemListStart}{\begin{itemize}}
\newcommand{\resumeItemListEnd}{\end{itemize}\vspace{-5pt}}

%-------------------------------------------
%%%%%%  RESUME STARTS HERE  %%%%%%%%%%%%%%%%%%%%%%%%%%%%


\begin{document}

%----------HEADING----------
% \begin{tabular*}{\textwidth}{l@{\extracolsep{\fill}}r}
%   \textbf{\href{http://sourabhbajaj.com/}{\Large Sourabh Bajaj}} & Email : \href{mailto:sourabh@sourabhbajaj.com}{sourabh@sourabhbajaj.com}\\
%   \href{http://sourabhbajaj.com/}{http://www.sourabhbajaj.com} & Mobile : +1-123-456-7890 \\
% \end{tabular*}

\begin{center}
    \textbf{\Huge \scshape Zac Todd} \\ \vspace{1pt}
    \small +64 027 5252 414 $|$ \href{zactodd0@gmail.com}{\underline{jake@su.edu}} $|$ 
    \href{https://linkedin.com/in/...}{\underline{linkedin.com/in/jake}} $|$
    \href{https://github.com/...}{\underline{github.com/jake}}
\end{center}


%-----------EDUCATION-----------
\section{Education}
  \resumeSubHeadingListStart
      \vspace{-2pt}
      \item
        \begin{tabular*}{0.97\textwidth}[t]{l@{\extracolsep{\fill}}r}
          \textbf{University of Canterbury} & Christchurch, NZ \\
          \textit{\small{Bachelor of Engineering with Honours in Software Engineering}} & \textit{\small{Feb. 2014 -- Dec. 2017}} \\
          \textit{\small{Masters in Applied Data Science}} & \textit{\small{Feb. 2018 -- Mar. 2019}} \\
          \textit{\small{Ph.D Computational and Applied Mathematics}} & \textit{\small{Jul. 2020 -- Present}} \\
        \end{tabular*}
        \vspace{-7pt}
  \resumeSubHeadingListEnd


%-----------EXPERIENCE-----------
\section{Experience}
  \resumeSubHeadingListStart

    \resumeSubheading
      {Research Assistant}{Nov. 2019 -- Present}
      {Spatial and Image Learning Group (SAIL), University of Canterbury}{Christchurch, NZ}
      \resumeItemListStart
        \resumeItem{Developed a REST API using FastAPI and PostgreSQL to store data from learning management systems}
        \resumeItem{Developed a full-stack web application using Flask, React, PostgreSQL and Docker to analyze GitHub data}
        \resumeItem{Explored ways to visualize GitHub collaboration in a classroom setting}
      \resumeItemListEnd
      
% -----------Multiple Positions Heading-----------
    \resumeSubSubheading
    {Waka Kotahi: Road Sign Classification, Detection and Segmentation using 360 Panoramic Images}{Nov. 2019 -- Jun. 2020}
    \resumeItemListStart
        \resumeItem{Developed a model to localise classify and segment road signs using 360 panoramic images.}
    \resumeItemListEnd
    \resumeSubSubheading
    {Waka Kotahi: Road Surfaces Extraction and Geometric Parameter Estimation using Lidar}{Nov. 2019 -- Jun. 2020}
    \resumeItemListStart
        \resumeItem{??? transformation method to convert lidar point clouds to colour image.}
        \resumeItem{Developed models to segment road lanes, segment and classify road markings, and segment road seal.}
        \resumeItem{Developed a method to estimate road seal width and road seal volume.}
    \resumeItemListEnd
    \resumeSubSubheading
    {Christchurch City Council (CCC): Road Defect Detection}{Jul. 2020 -- Jan. 2021}
    \resumeItemListStart
        \resumeItem{Collected 3000km of statehighway and urban roads consisting of 120k images.}
        \resumeItem{Developed a model to detect potholes and other defects with in an image.}
        \resumeItem{Build "Edge Capture and Processing" (ECAP) software product. ECAP provides automatic capture, processing and data upload. ECAP can capture with variety of sensor systems including lidar and depth cameras, for data processing ECAP can uses a variety of models built in either Tensorflow or PyTorch, and for data upload ECAP can upload to a varity of storgae solutions such as AWS, Dropbox, GDrive or remote serves using SSH. ECAP was developed using Python and ROS}
        \resumeItem{ECAP proof of concept; mounted Intel NUC running ECAP inside a CCC contractor's road sweeping vehicle, capture camera data and uploading to AWS. The proof of concept resulted in 20k image being collected and the detection of 8k defects and 30 potholes.}
    \resumeItemListEnd
    \resumeSubSubheading
    {Waka Kotahi: Road Asset Classification, Detection and Segmentation}{Nov. 2020 -- Present}
    \resumeItemListStart
        \resumeItem{Aided in the development of a model to localise, classify and segment road assets, these assets include road barriers, road signs, and ATPs.}
    \resumeItemListEnd
    \resumeSubSubheading
    {Statistics New Zealand: Urban Dwelling Detection using Satellite and Areal Imagery}{Nov. 2020 -- Present}
    \resumeItemListStart
        \resumeItem{Pre-processing of Sentinel satellite imagery and Land Information New Zealand (LINZ) areal imagery to fit GPU limitation.}
        \resumeItem{Aided in the development of models to detect dwellings within an urban area.}
    \resumeItemListEnd
    \resumeSubSubheading
    {Statistics New Zealand: Landcover Classification, Detection and Segmentation using Satellite Imagery}{Nov. 2020 -- Present}
    \resumeItemListStart
        \resumeItem{Pre-processing of Sentinel and LINZ satellite imagery.}
        \resumeItem{Aided in the development of models ro classify and segment different types of land cover such as different types of vegetation and water bodies.}
    \resumeItemListEnd
    \resumeSubSubheading
    {Waka Kotahi: Statehighway Data Collection Mobile Mapping Sensor Selection}{Dec. 2020 -- Present}
    \resumeItemListStart
        \resumeItem{Testing a sensors of sensors both external and internal mounting them on a mobile mapping vehicle.}
        \resumeItem{Field testing of sensors using ECAP.}
        \resumeItem{Collection of statehighway road data using lidar, depth cameras and standard cameras.}
        \resumeItem{Expanding ECAP to include sensor calibration}
    \resumeItemListEnd
    

%-----------PROJECTS-----------
\section{Projects}
    \resumeSubHeadingListStart
      \resumeProjectHeading
          {\textbf{Gitlytics} $|$ \emph{Python, Flask, React, PostgreSQL, Docker}}{June 2020 -- Present}
          \resumeItemListStart
            \resumeItem{Developed a full-stack web application using with Flask serving a REST API with React as the frontend}
            \resumeItem{Implemented GitHub OAuth to get data from user’s repositories}
            \resumeItem{Visualized GitHub data to show collaboration}
            \resumeItem{Used Celery and Redis for asynchronous tasks}
          \resumeItemListEnd
      \resumeProjectHeading
          {\textbf{Simple Paintball} $|$ \emph{Spigot API, Java, Maven, TravisCI, Git}}{May 2018 -- May 2020}
          \resumeItemListStart
            \resumeItem{Developed a Minecraft server plugin to entertain kids during free time for a previous job}
            \resumeItem{Published plugin to websites gaining 2K+ downloads and an average 4.5/5-star review}
            \resumeItem{Implemented continuous delivery using TravisCI to build the plugin upon new a release}
            \resumeItem{Collaborated with Minecraft server administrators to suggest features and get feedback about the plugin}
          \resumeItemListEnd
    \resumeSubHeadingListEnd



%
%-----------PROGRAMMING SKILLS-----------
\section{Technical Skills}
 \begin{itemize}[leftmargin=0.15in, label={}]
    \small{\item{
     \textbf{Languages}{: Java, Python, C/C++, SQL (Postgres), JavaScript, HTML/CSS, R} \\
     \textbf{Frameworks}{: React, Node.js, Flask, JUnit, WordPress, Material-UI, FastAPI} \\
     \textbf{Developer Tools}{: Git, Docker, TravisCI, Google Cloud Platform, VS Code, Visual Studio, PyCharm, IntelliJ, Eclipse} \\
     \textbf{Libraries}{: pandas, NumPy, Matplotlib}
    }}
 \end{itemize}

%-----------PUBLICATIONS----------
\section{Publications}

%-------------------------------------------
\end{document}
